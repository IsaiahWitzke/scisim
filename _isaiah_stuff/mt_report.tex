\documentclass[12pt]{article}
\usepackage{amsmath}
\usepackage{amssymb}
% \usepackage{kpfonts}
% \newcommand*{\vv}[1]{\vec{\mkern0mu#1}}
\newcommand{\GA}{G_{\mathbb{A}}}
\newcommand{\bigA}{{\mathbb{A}}}
\newcommand{\pd}[2]{\cfrac{\partial{#1}}{\partial{#2}}}
\newcommand{\dist}[2]{\text{dist(} #1 \text{, } #2 \text{)}}
\newcommand{\ball}[1]{\text{ball(} #1 \text{)}}
\newcommand{\proj}[2]{\text{proj}_{#1}(#2)}
\newcommand{\realm}[2]{ M_{#1 \times #2}(\mathbb{R})}
\begin{document}


\section{Preamble/Notation}

The primary problem posed in this project, is to find a solution to the following LCP:

find $\lambda \in \mathbb{R}^n$ such that...

$Q\lambda + b \geq 0$, $\lambda \geq 0$, $\lambda^T (Q\lambda + b) = 0$

With
$\mathbf{Q} := \GA^T M^{-1} \GA$,
$\mathbf{b} := \GA^T \dot{q}^- (1 + c_r)$


\section{Result: Relating Q matrix to angles}

What does $Q$ actually represent? Let's simplify for now and assume that $M = I = M^{-1}$
so that $Q = \GA^T M^{-1} \GA = \GA^T \GA$

$\GA \in M_{3m \times n}(\mathbb{R})$
where $n = $ number of collisions (size of "active set"),
and $m = $ the number of balls in the simulation.
Here we assume that each ball has 3 coordinates: an $x, y,$ and a rotation $\phi = 0$
\footnote{
    We aren't yet concerned with friction effects,
    so balls with no rotation initially stay that way during and after collision.
}
(hence $3m$ rows).

Every column of $\GA$ can be written as $(\GA^T)_i = \nabla g_i(q)$.
If $g_i(q)$ is the collision constraint between balls \textbf{a} and \textbf{b},
then, using $a_x, a_y, b_x, b_y$ to represent the indices of the x and y coordinates
of balls a and b respectively:

\begin{align*}
    g(q)    &= \dist{a}{b} \\
            % &= \dist{(q_{a_x}, q_{a_y})}{(q_{b_x}, q_{b_y})}\\
            &= \sqrt{(q_{a_x} - q_{b_x})^2 + (q_{a_y} - q_{b_y})^2} - r_a - r_b
\end{align*}

Where $r_a, r_b$ are the radii of balls a and b.\footnote{It's worth noting that this is
the ONLY place where the radius of the balls will appear. Once we take the gradient of $g(q)$
the $r$ constants will disappear.}

The constraint gradient (i.e. the columns that make up $\GA$) can now be written as:

\begin{align*}
    \nabla g_i(q)
        &= \begin{bmatrix}
            \dots \;
            \pd{\dist{a}{b}}{q_{a_x}} \; \pd{\dist{a}{b}}{q_{a_y}} \;
            \dots \;
            \pd{\dist{a}{b}}{q_{b_x}} \; \pd{\dist{a}{b}}{q_{b_y}} \;
            \dots \;
        \end{bmatrix}^T \\
        &= \begin{bmatrix}
            0 \dots  0 \;
            \pd{\dist{a}{b}}{q_{a_x}} \; \pd{\dist{a}{b}}{q_{a_y}} \;
            0 \dots 0 \;
            \pd{\dist{a}{b}}{q_{b_x}} \; \pd{\dist{a}{b}}{q_{b_y}} \;
            0 \dots 0 \;
        \end{bmatrix}^T \\
        &= \begin{bmatrix}
            0 \dots 0 \;
            \cfrac{q_{a_x} - q_{b_x}}{\dist{a}{b}} \; \cfrac{q_{a_y} - q_{b_y}}{\dist{a}{b}} \;
            0 \dots 0 \;
            \cfrac{q_{b_x} - q_{a_x}}{\dist{a}{b}} \; \cfrac{q_{b_y} - q_{a_y}}{\dist{a}{b}} \;
            0 \dots 0 \;
        \end{bmatrix}^T
\end{align*}

Notice that the sub-vector
$\vec{n_{ab}}
    := \begin{bmatrix}
        \pd{\dist{a}{b}}{q_{a_x}} \; \pd{\dist{a}{b}}{q_{a_y}}
    \end{bmatrix}^T
    = \begin{bmatrix}
        \cfrac{q_{a_x} - q_{b_x}}{\dist{a}{b}} \; \cfrac{q_{a_y} - q_{b_y}}{\dist{a}{b}}
    \end{bmatrix}^T$
which is made up of the first 2 non-zero values of $\nabla g_i(q)$
is the collision normal vector $\vec{n_{ba}}$!
Similarly, the other 2 non-zero entries of $\nabla g_i(q)$ make up the opposing 
collision normal: $-\vec{n_{ba}} = \vec{n_{ab}}$

Now we know what the columns of $\GA$ consist of, we can look at the individual elements of $Q$:

\begin{align*}
Q_{ij}
    &= \GA^T{_i} \cdot \GA^T{_j}\\
    &= \nabla g_i(q) \cdot \nabla g_j(q)\\
    &= ...
\end{align*}

\subsection*{... 3 Cases} 
\subsubsection*{1: $i = j$ (2 balls)} 

In this case, $Q_{ij} = Q{ii} = \nabla g_i(q) \cdot \nabla g_i(q) = ||\vec{n_{ab}}||^2 + ||\vec{n_{ba}}||^2 = 2$

\subsubsection*{2: $i \neq j$ with 2 separate collisions (4 balls)}

Consider $Q_{ij} = \nabla g_i(q) \cdot \nabla g_j(q)$
where $g_i(q) = \dist{a}{b}$ and $g_j(q) = \dist{c}{d}$. 
i.e. we consider 2 collisions (i,j) involving 4 distinct balls (a,b,c,d).
Here the 4 non-zero entries of
$g_i(q)$ will occur at different indices than the non-zero elements of $g_j(q)$!
\footnote{The $k$th element of $\nabla g_i(q) \neq 0$ means that the $k$th element
is the partial of $g$ w.r.t. either ball a or b. This implies that the $k$th element
of $\nabla g_j(q) = \pd{g_j(q)}{q_k} = 0$ since $k \in \{a_x, a_y, b_x, b_y\}$ and constraint between balls c and d is independent
of a or b's position.}
Therefore: $Q_{ij} = \nabla g_i(q) \cdot \nabla g_j(q) = 0$

\subsubsection*{3: $i \neq j$ with 2 interacting collisions (3 balls)}

The most interesting case! Let's assume WLOG that 3 balls a, b, and c are colliding simultaneously.
WLOG, assume $g_i(q) = \dist{a}{b}$, and $g_j(q) = \dist{b}{c}$ so that $i, j \in \mathbb{A}$.

Then
$Q_{ij}
    = \nabla g_i(q) \cdot \nabla g_j(q)
    = \vec{n_{ab}} \cdot \vec{n_{cb}}
    = |\vec{n_{ab}}||\vec{n_{cb}}|\cos{\theta}
    = \cos{\theta}
$

Where $\theta := \angle abc$

\section*{Result: interpretation of $b$, $Q \lambda$, LCP criteria}

\subsection*{$b$}

$b$ is the other constant value of interest in out LCP. If we assume once again that
$g_i(q)$ is the constraint between balls a and b:

\begin{align*}
    b   &:= \GA^T \dot{q}^-(1 + c_r)\\
    \implies b_i &= (1 + c_r) \nabla g_i(q) \dot{q}^-\\
    &= (1 + c_r) \frac{dg_i}{dt}\\
    &= (1 + c_r) \frac{d(\dist{a}{b})}{dt}\\
\end{align*}
Interpretation: the $b_i$ represents\\
\indent $\mathbf{-(1+c_r)} \times$ (relative speed ball a is approaching ball b).\\
(note the negative sign since $\frac{d(\dist{a}{b})}{dt} \leq 0$)

\subsection*{$Q \lambda$}

From the Rosi paper, we know that $\lambda \in \mathbb{R}^{|\mathbb{A}|}$ is the vector of
impulse coefficients (i.e. $\lambda_i$ is the force that ball a exerts on ball b
***TODO: ensure this is *technically/semantically* correct!)
Now consider the $i$th element of $Q\lambda$:

\begin{align*}
(Q \lambda)_i 
    &= \sum_{j = 1}^{\bigA} Q_{ij} \lambda_j\\
\end{align*}

For $Q_{ij}$ there are 3 cases (see above). Firstly, collision constraint $i$ could
not be affected by either of the balls involved in collision $j$, in which case $Q_{ij} = 0$
and we can ignore those terms. Secondly, there will be a term where $i = j$, in this case
$Q_{ij} = Q_{ii} = 2$. And finally, assume that constraint $i$ is "concerned" with balls a and b,
(i.e. $g_i(q) = \dist{a}{b}$) and $j$ is "concerned" with \textit{either}
ball a or b \textit{and} some 3rd ball c. In this case
$Q_{ij} = \cos(\angle abc)$ or $Q_{ij} = \cos(\angle cab)$.

So, if we set:\\
$A = \{x \in \bigA | s.t. \; g_x(q) = \dist{a}{c}\} \backslash \{i\}$ \\
And $B = \{x \in \bigA | s.t. \; g_x(q) = \dist{b}{c}\} \backslash \{i\}$  \\
Where in both cases c is just a ball that is in the process of colliding with ball a or b respectively, then
we can rewrite above as:

\begin{align*}
(Q \lambda)_i 
    &= \sum_{j = 1}^{\bigA} Q_{ij} \lambda_j\\
    &= 2\lambda_i + \sum_{x \in A}^{} \cos(\angle ba\;\ball{x})\lambda_x + \sum_{x \in B}^{} \cos(\angle ab\;\ball{x})\lambda_x\\
    &= 2\lambda_i
        + ||\sum_{x \in A}^{}\proj{\overrightarrow{n_{ba}}}{\overrightarrow{n_{\ball{x}a}}} \lambda_x||
        + ||\sum_{x \in B}^{}\proj{\overrightarrow{n_{ab}}}{\overrightarrow{n_{a\;\ball{x}}}} \lambda_x||\\
    &= \text{net force acting on balls a and b along direction: } \overrightarrow{n_{ab}}
\end{align*}


TODO: clean up the equation above... using "$\ball(x)$" is confusing... maybe some of the ab should be ba...
theres a little more explaining that could be done. ESPECIALLY relating to our disregard of mass - 
really, the above should be the net $\Delta \dot{q}$ once we bring back $M^{-1}$ and divide each
term by it's balls' masses.

\subsection*{LCP Criteria: $Q\lambda + b \geq 0$}

This is saying for each collision $i \in \bigA$, we need $(Q\lambda)_i \geq -b_i$.
Let $[\dot{q}]_{\overrightarrow{n_{ab}}}$ represent the speed of ball a wrt ball b.\\
As we have seen before, $b_i = -(1 + c_r) [\dot{q}]_{\overrightarrow{n_{ab}}}$
so we can rewrite our lcp condition as:

\begin{align*}
    \text{net force (speed?) acting on balls a and b} \geq (1 + c_r) [\dot{q}]_{\overrightarrow{n_{ab}}}
\end{align*}
\newline{}
Or something like that... basically, the LCP condition is enforcing our solution ($\lambda$,
the impulse coefficients) will result in exiting velocities that conserve momentum.




OHHH SHIT OKOKOKOKOK

when we factor in the complementary condition, the times 2 above makes sense!

either $(Q\lambda + b)_i = 0$, in this case $\lambda_i > 0$ and we get

\begin{align*}
(Q \lambda)_i &= \sum_{j = 1}^{\bigA} Q_{ij} \lambda_j\\
    &= 2\lambda_i + \sum_{x \in A}^{} \cos(\angle ba\;\ball{x})\lambda_x + \sum_{x \in B}^{} \cos(\angle ab\;\ball{x})\lambda_x\\
    &= 2\lambda_i
        + ||\sum_{x \in A}^{}\proj{\overrightarrow{n_{ba}}}{\overrightarrow{n_{\ball{x}a}}} \lambda_x||
        + ||\sum_{x \in B}^{}\proj{\overrightarrow{n_{ab}}}{\overrightarrow{n_{a\;\ball{x}}}} \lambda_x||\\
    &= ||\sum_{x \in A}^{}\proj{\overrightarrow{n_{ba}}}{\overrightarrow{n_{\ball{x}a}}} \lambda_x||
        + ||\sum_{x \in B}^{}\proj{\overrightarrow{n_{ab}}}{\overrightarrow{n_{a\;\ball{x}}}} \lambda_x||\\
\end{align*}

basically, our LCP solution in this case requires that the relative exit velocity of balls
a and b is ONLY a result of the forces of the OTHER balls colliding with balls a/b.

Or... wait no nvm lol

TODO: more investigation here perhaps?

\subsection*{result: Q is not generally an "M-matrix"}

This was something we were concerned with in previous semesters of work.
The off-diagonal entries of $Q$ are $\cos \theta$ where $\theta \in [0, \pi]$!
It's easy to imagine a scenario where 2 balls (a and b) 
simultaneously collide with a common 3rd ball (c) so that the angle
connecting the center of the 3 balls $\angle acb < \pi / 2$
and thus $\cos \theta > 0$, ($\implies Q$ cannot be an M-matrix )

TODO: illustration?

\section{overlapping and collisions through balls}

\section{"K3" Example}

Generally speaking, when only 2 balls collide with each other, there isn't much
special going on... $G_\bigA \in \realm{3m}{1}$ so $Q \in \realm{1}{1}$ and $\lambda$ is
really easy to find.

A more interesting scenario can be found when 3 balls collide with each other while all on the same axis.
\footnote{
    This does required that ball collides with ball c \textit{through} ball b... but sadly, because of discrete
    time issues, this is a case that must be considered
}
In this case, all angles are either $0$ or $\pi$, so $Q$ is:
\footnote{Order of values also depend on the order of collisions in the active set.}

\begin{align*}
Q
    &= \begin{bmatrix}
        2 & 1 & -1\\
        1 & 2 & 1\\
        -1 & 1 & 2
    \end{bmatrix}
\end{align*}

Which is singular! There are an infinite number of solutions to the LCP
- and IPOPT does in fact find a different solution to policy iteration!

\section{"K4" Example}

4 balls (all overlapping each other) configured in a square with velocities
towards the center of the square makes a "K4" type of graph where each of the 4 balls
is colliding with the other 3 in the same instant.
This also produces a singular Q matrix (rank = 5, nullity = 1), and
IPOPT/PI yield different solutions.

in both K3 and K4 examples, there are 3 viable control sets.
\end{document}